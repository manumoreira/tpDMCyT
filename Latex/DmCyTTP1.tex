\documentclass[conference]{IEEEtran}
\IEEEoverridecommandlockouts
% The preceding line is only needed to identify funding in the first footnote. If that is unneeded, please comment it out.
\usepackage{cite}
\usepackage{amsmath,amssymb,amsfonts}
\usepackage{algorithmic}
\usepackage{graphicx}
\usepackage{textcomp}
\usepackage{xcolor}
\usepackage{fancyhdr}
\def\BibTeX{{\rm B\kern-.05em{\sc i\kern-.025em b}\kern-.08em
    T\kern-.1667em\lower.7ex\hbox{E}\kern-.125emX}}
    
% ANCS'19 Conference Name %%%%%%%%%%%%%%%%%%%%%%%%%%%%%%%%%%%%%%%%%%%%%%%%%%%%%%%%%%%%%%%%
\chead{\rmfamily\fontsize{9}{30}\selectfont 
2025 Data Mining en Ciencia y Tecnología - Maestría en data mining FCEN UBA)}
\renewcommand{\headrulewidth}{0pt}
%%%%%%%%%%%%%%%%%%%%%%%%%%%%%%%%%%%%%%%%%%%%%%%%%%%%%%%%%%%%%%%%%%%%%%%%%%%%%%%%%%%%%%%%%%

% ANCS'19 Copyright Notice %%%%%%%%%%%%%%%%%%%%%%%%%%%%%%%%%%%%%%%%%%%%%%%%%%%%%%%%%%%%%%%%%%%%%%%%%%%%%%%%%%%%%%%%%%%%%%%%%
% Note: use the copyright text that matches your requirements.
%
% 1. For papers in which all authors are employed by the US government, the copyright notice is: 
%\cfoot{\rmfamily\fontsize{9}{0}\selectfont U.S. Government work not protected by U.S. copyright}
%
% 2. For papers in which all authors are employed by a Crown government (UK, Canada, and Australia), the copyright notice is: 
%\cfoot{\rmfamily\fontsize{9}{0}\selectfont 978-1-7281-4387-3/19/\$31.00~\copyright~2019 Crown}
%
% 3. For papers in which all authors are employed by the European Union, the copyright notice is:
%\cfoot{\rmfamily\fontsize{9}{0}\selectfont 978-1-7281-4387-3/19/\$31.00~\copyright~2019 European Union}
%
% 4. For all other papers the copyright notice is:
\cfoot{\rmfamily\fontsize{9}{0}\selectfont 978-1-7281-4387-3/19/\$31.00~\copyright~2019 IEEE}
%%%%%%%%%%%%%%%%%%%%%%%%%%%%%%%%%%%%%%%%%%%%%%%%%%%%%%%%%%%%%%%%%%%%%%%%%%%%%%%%%%%%%%%%%%%%%%%%%%%%%%%%%%%%%%%%%%%%%%%%%%%%%

\begin{document}

\title{TP1 Grupo 6
 o Una exploración sobre la robustez en redes humanas\\
{\footnotesize \textsuperscript{}}
}

\author{\IEEEauthorblockN{Serena Dituro}
\IEEEauthorblockA{\textit{dept. name of organization (of Aff.)} \\
\textit{name of organization (of Aff.)}\\
serenadituro@gmail.com}
\and
\IEEEauthorblockN{Lucas}
\IEEEauthorblockA{\textit{dept. name of organization (of Aff.)} \\
\textit{name of organization (of Aff.)}\\
email address}
\and
\IEEEauthorblockN{Manuel Moreira}
\IEEEauthorblockA{\textit{dept. name of organization (of Aff.)} \\
\textit{name of organization (of Aff.)}\\
manuelmoreira.ar@gmail.com}
\and
\IEEEauthorblockN{Joaquín Kalmbach}
\IEEEauthorblockA{\textit{dept. name of organization (of Aff.)} \\
\textit{name of organization (of Aff.)}\\
kalmbach.jr@gmail.com}
}

\maketitle
\thispagestyle{fancy}

\begin{abstract}
En este trabajo nos proponemos explorar algunas características topológicas de las redes. Vamos a enfocarnos en dos ejemplos de redes, una construida a partir de los personajes de la famosa novela de Victor Hugo, Los Miserables y otra proveniente de una base de datos de emails de una institución europea de investigación.
 
\end{abstract}

\begin{IEEEkeywords}
redes, grafos, robustez, centralidad, comunidades
\end{IEEEkeywords}

\section{Introduction}

Las redes elegidas tienen algunos puntos de coincidencia y varias diferencias, empezaremos por describirlas brevemente antes de entrar en comparaciones más técnicas.
La red de los miserables es una red pequeña de 77 nodos y 254 enlaces está connstruida a partir de la copresencia de personajes en los capítulos de la novela, al ser una red pesada el peso de cada enlace surje de la frecuencia de co-ocurrencia, al solo medir co-ocurrencia los enlaces son no-dirigidos. Es una red muy utilizada en el 'ambito de la ensenanza de análisis redes, pueden encontrarse numerosas publicaciones que utilizan esta red como insumo (citar) 
La otra red con la que vamos a trabajar es la red de Emails-EU, que también es una red muy famosa en el ámbito de análisis de redes cuenta con numerosas publicaciones que la utilizan como fuente (citar). Esta red está construida a partir del intercambio de emails entre integrantes de una institución de investigación de la unión europea, los lazos no tienen peso ni dirección solamente dan cuentaa de que existió un intercambio de emails entre dos integrantes que están representados como nodos.


\section{Métodos}

\subsection{Métricas de redes}
Lorem ipsum dolor sit amet, consectetur adipiscing elit. Donec vel egestas dolor, nec dignissim metus. Donec augue elit, rhoncus ac sodales id, porttitor vitae est. Donec laoreet rutrum libero sed pharetra.
\section{Resultados y discusión}
Lorem ipsum dolor sit amet, consectetur adipiscing elit. Donec vel egestas dolor, nec dignissim metus. Donec augue elit, rhoncus ac sodales id, porttitor vitae est. Donec laoreet rutrum libero sed pharetra.
\subsection{La robustez en redes humanas}
Lorem ipsum dolor sit amet, consectetur adipiscing elit. Donec vel egestas dolor, nec dignissim metus. Donec augue elit, rhoncus ac sodales id, porttitor vitae est. Donec laoreet rutrum libero sed pharetra.
\subsection{Lalala}
\begin{itemize}
\item Lorem ipsum dolor sit amet, consectetur adipiscing elit. Donec vel egestas dolor, nec dignissim metus. Donec augue elit, rhoncus ac sodales id, porttitor vitae est. Donec laoreet rutrum libero sed pharetra.
\item ALorem ipsum dolor sit amet, consectetur adipiscing elit. Donec vel egestas dolor, nec dignissim metus. Donec augue elit, rhoncus ac sodales id, porttitor vitae est. Donec laoreet rutrum libero sed pharetra.
\item Lorem ipsum dolor sit amet, consectetur adipiscing elit. Donec vel egestas dolor, nec dignissim metus. Donec augue elit, rhoncus ac sodales id, porttitor vitae est. Donec laoreet rutrum libero sed pharetra.
\end{itemize}

\section{Conclusiones}
Lorem ipsum dolor sit amet, consectetur adipiscing elit. Donec vel egestas dolor, nec dignissim metus. Donec augue elit, rhoncus ac sodales id, porttitor vitae est. Donec laoreet rutrum libero sed pharetra.

\section{Referencias}
\subsection {Les Miserables}
\begin{itemize}
    \item https://towardsdatascience.com/les-miserables-social-network-analysis-using-marimo-notebooks-and-the-networkx-python-library-%EF%B8%8F-%EF%B8%8F-3f433216412f/
    \item http://konect.cc/networks/moreno_lesmis/
    \item https://studentwork.prattsi.org/infovis/visualization/les-miserables-character-network-visualization/
    \item https://medium.com/@leviathan36/les-mis%C3%A9rables-why-not-study-its-storyline-with-graph-theory-5cf45f2473f9
\end{itemize}
\subsection{Emails-EU}
\begin{itemize}
    \item https://networkrepository.com/ia-email-EU.php
    \item https://publications.jrc.ec.europa.eu/repository/handle/JRC118027
    \item https://paperswithcode.com/dataset/email-eu (hay 33 papers que usan esta base)
    \item https://www.researchgate.net/publication/330163019_An_Analysis_of_Email-Eu-Core_Network

\end{itemize}




\begin{thebibliography}{00}
\bibitem{b1} G. Eason, B. Noble, and I. N. Sneddon, ``On certain integrals of Lipschitz-Hankel type involving products of Bessel functions,'' Phil. Trans. Roy. Soc. London, vol. A247, pp. 529--551, April 1955.
\bibitem{b2} J. Clerk Maxwell, A Treatise on Electricity and Magnetism, 3rd ed., vol. 2. Oxford: Clarendon, 1892, pp.68--73.
\bibitem{b3} I. S. Jacobs and C. P. Bean, ``Fine particles, thin films and exchange anisotropy,'' in Magnetism, vol. III, G. T. Rado and H. Suhl, Eds. New York: Academic, 1963, pp. 271--350.
\bibitem{b4} K. Elissa, ``Title of paper if known,'' unpublished.
\bibitem{b5} R. Nicole, ``Title of paper with only first word capitalized,'' J. Name Stand. Abbrev., in press.
\bibitem{b6} Y. Yorozu, M. Hirano, K. Oka, and Y. Tagawa, ``Electron spectroscopy studies on magneto-optical media and plastic substrate interface,'' IEEE Transl. J. Magn. Japan, vol. 2, pp. 740--741, August 1987 [Digests 9th Annual Conf. Magnetics Japan, p. 301, 1982].
\bibitem{b7} M. Young, The Technical Writer's Handbook. Mill Valley, CA: University Science, 1989.
\end{thebibliography}
\vspace{12pt}
\color{red}
IEEE conference templates contain guidance text for composing and formatting conference papers. Please ensure that all template text is removed from your conference paper prior to submission to the conference. Failure to remove the template text from your paper may result in your paper not being published.

\end{document}
